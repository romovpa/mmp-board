%%% DOCUMENT
\documentclass[a4paper]{article}
% article options:
%   - a4paper, letterpaper
%   - twocolumn
%   - landscape
%   - 10p, 11pt, 12pt
%   - draft

%%% PACKAGES
\usepackage[T2A]{fontenc}
\usepackage[utf8]{inputenc} 
\usepackage[pdftex,unicode]{hyperref}
\usepackage[russian,english]{babel}   % language

\usepackage{amsmath,amssymb}
\usepackage{graphicx}
\usepackage{multicol}
\usepackage{subfig}
\usepackage{wrapfig}
\usepackage{float}
\usepackage{color}
\usepackage{pb-diagram}
%\usepackage{movie15}

% style
\usepackage{indentfirst}
\usepackage{fullpage}

%%% TITLE
\title{Искусственный интеллект для игры "Коньки"}
\author{Петр Ромов, группа 317}
\date{}

%%% CONTENT
\begin{document}

\maketitle

\section{Описание игры}

Игра происходит на доске размера $8 \times 8$. На доске изначально находятся фишки двух цветов: красные и синие, некоторые клетки доски помечены цифрами (1 или 2). Начальный счет 0--0. Игра ведется по следующим правилам:
\begin{enumerate}
\item Игроки ходят поочередно: первый игрок --- красными фишками, второй --- синими. За один ход игрок обязан передвинуть одну фишку.
\item Фишки ходят как шахматный конь.
\item Ходить можно на любую клетку, не занятую своей фишкой.
	\begin{itemize}
	\item Если клетка была помечена цифрой, метка удаляется, сделавшему ход прибавляется соответствующее цифре число очков.
	\item Если клетка была занята фишкой соперника, то фишка соперника ``съедается'', а сделавшему ход прибавляется 1 очко.
	\end{itemize}
\item Если фишки одного из игроков располагаются на одной горизонтали/вертикали, то этому игроку прибавляется 3 очка, игра заканчивается. При этом, если после некоторого хода у обоих игроков фишки располагаются на одной горизонтали/вертикали, то обоим игрокам прибавляется 3 очка.
\item Выигрывает игрок, набравший наибольшее число очков. При равенстве очков --- ничья.
\item Если за 30 ходов игра не закончилась, засчитывается ничья (независимо от текущего счета).
\end{enumerate}

\section{Алгоритм и реализация}

\section{Примеры работы алгоритма}

\section{Выводы}

\end{document}
